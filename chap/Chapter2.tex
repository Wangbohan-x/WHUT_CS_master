\chapter{这是第二章}

\section{伪代码}
如需阐述算法看用伪代码形式展示,如算法 \ref{al:forward_prop} 所示。

\begin{algorithm}[H]
    \SetAlgoNoLine % 不要算法中的竖线
    \SetKwInOut{Input}{\textbf{输入}}\SetKwInOut{Output}{\textbf{输出}} % 替换关键词
    \Input{
        \\
        训练样本$x$(对应输入层输出$a^0$),标签$y$;\\
        网络权重$\{W^l\}$,网络偏置$\{b^l\}$;\\
        激活函数$\sigma$,神经网络总层数$L$;\\
    }
    \Output{
        \\
        各层的$z^l$和$a^l$(用于后续反向传播);\\
        损失值$\text{Loss}$(可选);\\
    }
    \BlankLine
    初始化输入层输出$a^0 = x$;\\ % 分号 \; 可保留,此处贴合模板用\\
    遍历隐藏层和输出层($l$从1到$L$);\\
    \For{$l = 1$ \textbf{to} $L$}{
        计算第$l$层加权输入$z^l = W^{(l-1)} \cdot a^{(l-1)} + b^{(l-1)}$($\cdot$表示矩阵乘法);\\
        计算第$l$层激活输出$a^l = \sigma(z^l)$($\sigma$为Sigmoid/ReLU/Softmax等);\\
    }
    计算损失函数值$\text{Loss} = \mathcal{L}(a^L, y)$;\\
    返回各层$z^l$、$a^l$和损失值$\text{Loss}$;\\
    \caption{前向传播确定网络各层输出\label{al:forward_prop}}
\end{algorithm}

\section{实用工具}
VSCcode + LatexWorkshop插件,强大的编辑和编译功能。
\begin{figure}[H]
\centering
\includegraphics[scale=0.55]{figure/chap2/2-1.png}
\caption{LatexWorkshop插件}
\label{fig2-1}
\end{figure}
Google Scholar,强大的文献搜索引擎。可直接导出.bib格式文献。
\begin{figure}[H]
\centering
\includegraphics[scale=0.3]{figure/chap2/2-2.png}
\caption{Google Scholar}
\label{fig2-2}
\end{figure}

NotebookLLM,多模态大模型,。支持文档问答、代码生成等功能。
\begin{figure}[H]
\centering
\includegraphics[scale=0.3]{figure/chap2/2-3.png}
\caption{NotebookLLM}
\label{fig2-3}
\end{figure}
\section{参考文献引用}
上角标引用参考文献像这样\cite{liu2025cage},方括号引用,文献\parencite{liu2025cage}。
